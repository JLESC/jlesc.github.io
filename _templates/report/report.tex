\documentclass[12pt]{article}

\usepackage{times}
\usepackage[utf8]{inputenc}
\usepackage[american]{babel}
\usepackage{fontenc}
\usepackage{graphicx}
\usepackage{amsfonts,amsthm,amsmath,amssymb}
\usepackage{xspace}
\usepackage{todonotes}
\usepackage{fullpage}
\usepackage{paralist}
\usepackage{mdwlist}
\usepackage{pdfpages}
\usepackage{titling}
\usepackage{longtable}
\usepackage{booktabs}

\usepackage[backend=biber]{biblatex}

%collaboration symbols
\usepackage{color}
\definecolor{darkgreen}{rgb}{0.09, 0.45, 0.27}
\newcommand{\ema}[1]{\ensuremath{#1}\xspace}
\newcommand{\ongoing}{{\Large \color{darkgreen}{\ema{\rightrightarrows}}\xspace}}
\newcommand{\starting}{{\Large \color{green}{\ema{| \hspace*{-0.3cm}\rightrightarrows}}\xspace}}
\newcommand{\finished}{{\Large \color{blue}{\ema{\rightrightarrows \hspace*{-0.3cm} |}}\xspace}}
%\newcommand{\blocked}{{\Large \color{red}{\ema{\nRightarrow}}\xspace}}
\newcommand{\blocked}{{\Large \color{red}{\ema{\downdownarrows}}\xspace}}
%\newcommand{\explore}{{\Large \color{black}{\ema{\rightrightarrows}}\xspace}}
\newcommand{\explore}{{\Large \color{magenta}{\ema{\upuparrows}}\xspace}}
%\theoremstyle{theorem}
\newtheorem{theorem}{Theorem}
\newtheorem{lemma}{Lemma}
\newtheorem{proposition}{Proposition}
\newtheorem{property}{Property}
\theoremstyle{definition}
\newtheorem{definition}{Definition}

\pretitle{
  \vspace*{-2cm}
  \begin{center}
  \includegraphics[width=\textwidth]{jlesc_logo.png}\\[2\bigskipamount]
  \LARGE
}
\posttitle{\end{center}}

\title{JLESC -- Activity Report 2015}

\author{Franck Cappello, William Kramer, Jesus Labarta, Yves Robert, Robert Speck}

\date{June 2015}


\begin{document}

\maketitle
\thispagestyle{empty}
\vspace*{0.5cm}

\section*{Executive summary}

We present here a brief executive summary of the 2014-2015 activities and some of the planned activities for 2015-2016:

\begin{itemize}
\item Total number of projects: 16
    \vspace{-0.5\baselineskip}
    \begin{itemize*}
        \item Starting projects: 8
        \item Ongoing projects: 8
    \end{itemize*}
\item Total number of persons involved: 51
    \vspace{-0.5\baselineskip}
    \begin{itemize*}
        \item Permanent personnel: 26
        \item Students and postdocs: 25 
    \end{itemize*}
\item Total number of Person Months: 127
\item Number of visits: 7, number of visit months: 26.5
\item Number of planned visits: 11, number of planned visit months: 22.75
\item Number of joint publications (accepted or submitted): 12 (4 on finished activities)
\item Number of new software prototypes/software improvements: 3
\item Number of defended Ph.Ds on collaborative activities: 2
\item Number of additional funding: 1 
\item Number of submitted international proposals: 1
\item Number of student awards: 1
\end{itemize}


\newpage

\tableofcontents

\newpage
\section{Introduction}
\label{sec.intro}

\subsection{Purpose} 
The purpose of the Joint Laboratory for Extreme Scale Computing (JLESC) is to be an international, virtual organization whose goal is to enhance the ability of member organizations and investigators to make the bridge between Petascale and Extreme computing, following the model of the Joint Laboratory on Petascale Computing (JLPC). The founding partners of the JLESC are Inria and UIUC. JLESC will involve computer scientists, engineers and scientists from other disciplines as well as from industry, to ensure that the research facilitated by the Laboratory addresses science and engineering's most critical needs and takes advantage of the continuing evolution of computing technologies.

\subsection{Term and Termination} 
The JLESC will last four (4) years, from the date of signature of the agreement (June 2014). After the original time period, the agreement can be renewed if the majority of full partners agree by  written amendment. The Agreement may only be revised by written consent of the Founding Partners.
The Agreement may be terminated at any time by the Founding Partners mutual consent in writing. Alternatively, a Partner that wishes to terminate its participation in the JLESC will provide ninety (90) days' notice of such termination.

\subsection{Objectives} 
The objectives of the JLESC are to initiate and facilitate international collaborations on research and state of the practice topics, related to computational and data focused simulation and analytics at scale. The JLESC will facilitate the production of original ideas, publications, discussion forums, research reports, products and open source software, aimed to address the most critical issues in advancing from petascale to extreme scale computing.

\subsection{Research Topics, Activities and Joint Projects} 
JLESC research topics include: parallel programming models and libraries, numerical algorithms and libraries, parallel I/O systems and libraries, data analytics, graph algorithms, heterogeneous computing, resilience, system and application performance analysis and modeling, storage infrastructure design and efficiency and productivity tools.  Research topics may evolve over the course of the JLESC existence.
JLESC \textbf{Joint Projects} are defined and tracked by the JLESC.  A Joint Project will have at least two co-Principal Investigators, each from the different JLESC partner institutions involved in the project, a statement of goals for the project, a program of work, and a  schedule to accomplish the goals.  The activities funded by JLESC will include:
\begin{compactenum}
\item A 2 or 3 day workshop every six months with all JLESC partners. The purpose of these workshops is to discuss the work performed in the joint projects during the last six months and to jointly plan work for the next six months. 
\item Researcher/student exchanges. These exchanges take the form of short and long term visits (from days to 1 year or more) between staff and students of JLESC institutions, at different JLESC partner sites, in order to work on joint projects. If required by host institution, separate agreements will be signed between the visiting and host institution specifying the terms of the exchange.
\end{compactenum}

It is expected the number of staff participating and the degree of contributions will be approximately the same across all partner institutions.

\subsection{Partnership}
JLESC will have three levels of partners: \textbf{Founding Partners}, \textbf{Full Partners} and \textbf{Associate Partners}.   
The Founding Partners are Inria and UIUC and are considered as initial Full Partners. 

Full Partners participate in the JLESC management, have the ability to attend workshops and JLESC functions; will pose and co-lead collaborative workshops, will commit certain levels of funding to the JLESC collaboration for specified time periods; will contribute to setting the priorities of JLESC; and will provide access to facilities at their institution (if available).  
Associate Partners are those entities who intend to become Full Partners but are not yet meeting the obligations of full partnership as described in this document.  
The maximum period an institution can be an Associate Partner is twelve months.  After that period, the institution either reaches full partnership or leaves the JLESC. The Founding Partners may agree to extend the period of associate partnership with good cause.
In order to retain active status as a JLESC partner, institutions must fully participate in JLESC activities. Any institution that has not fully and actively participated in at least two JLESC activities over an eighteen (18) month span will be considered for having its partnership revoked. The current members of the JLESC are:
\begin{compactenum}
\item UIUC/NCSA (Founding partner)
\item Inria (Founding partner)
\item ANL (Full partner)
\item BSC (Full partner)
\item JSC (Full partner)
\item Riken (Full partner)
\end{compactenum}

\subsection{Funding Model}
Each Full Partner institution will provide funding dedicated to support the two types JLESC activities described above. The mechanism is that each partner institution allocates an annual budget that is managed locally (for ease of administration) both by the local representative and the JLESC director.  Each Full Partner's contributions dedicated to JLESC activities will be approximately equal in value.  Collaboration effort costs, staff costs, or in-kind costs are not considered dedicated funding.  
Full Partnership will be executed by signing the statement in Attachment A and recording it with the Founding Partners.  Attachment A states the new Full Partner undertakes the rights and obligations of the agreement.  The new partner is considered a Full Partner when Attachment A is signed by an authorized representative of the new partner and the chairman of the JLESC steering committee.  
Due to the US Department of Energy Policy, the Founding Partners agree to a one time exception for Argonne National Laboratory (ANL). 

ANL's Full Partnership membership is based on a letter of commitment from Dr. Marc Snir to carry out all Full Partner responsibilities as an employee of ANL and as defined in this agreement, including equivalent funding of JLESC activities and responsibilities.
The level of JLESC funding from each Full Partner will be set by agreement of the Founding Partners.
Associate Partners shall allocate an annual budget that is managed locally and controlled both by the local representative and the JLESC director, but would be less than the funding levels provided by Full Members. 
Associate Partners may also directly cover the full cost of their staff participation in JLESC activities, including paying participation fees for attending JLESC activities. In the latter case, the levels of funding should be sufficient for equivalent participation in the JLESC activities.  Collaboration costs, staff costs, or in-kind costs are not considered in determining an equivalent value.  Associate Partnership will be indicated by providing a "letter of commitment" to participate in JLESC activities to the extent the partner adds observable value to the JLESC.
For visiting activities, whatever is the selected funding model, a researcher from institution A, visiting institution B will only be supported by funds provided by institutions A and B. 
Each party involved in the collaboration will cover its personnel salaries and associated costs. The home institution(s) of the director shall cover the director's salary.
There will be approximately one-year overlap of the JLPC agreement between NCSA and Inria and the JLESC agreement.  For the sake of this agreement, any remaining JLPC funding at the expiration of the JLPC agreement will transition to JLESC funding. 

\subsection{Intellectual Property}
Whenever and as often as possible, the partners will produce open publications and results, with the organizations participating in the collaborative effort jointly deciding on the intellectual property rights for the work carried out by the partners.  
For any results that require protection of Intellectual Property, the institutions participating in the development of that result will jointly resolve it between themselves by developing specific agreements for covering those activities and IP rights.

\subsection{Management and Governance} 
\enlargethispage{\baselineskip}
The \textbf{Steering Committee} will consist of the heads (or their representative) of the involved Full Partner institutions. The Steering Committee is responsible for the overall success of JLESC activities, for ensuring that the JLESC has sufficient participation from their respective institutions and for providing strategic guidance to the Director and Executive committee. 
The Steering Committee shall self-select a chairperson once a year. All decisions should be taken unanimously. If a unanimous decision cannot be reached, Founding Partners will determine the final decision amicably, in a spirit of cooperation and goodwill. The current members of the JLESC steering committee are:
\begin{compactenum}
\item Ed. Seidel, Chair of the Steering Committee (UIUC/NCSA)
\item Antoine Petit (Inria)
\item William Harrod (DoE)
\item Mateo Valero (BSC)
\item Thomas Lippert (JSC)
\item Hirao Kimihiko (Riken)
\end{compactenum}
\bigskip

A \textbf{Scientific Advisory Committee} will assess the progress and the output of JLESC and suggest/recommend evolutions/adjustments on strategic and financial questions. The Scientific Advisory Committee will be composed of external researchers and experts of JLESC research topics and will provide evaluation, advice and recommendations to the JLESC Steering  Committee,  Director and Executive Committee in a written report. Scientific Advisory Committee members will be recruited from non-partner entities. The current members of the JLESC Scientific Advisory committee are:
\begin{compactenum}
\item Miron Livny (University of Wisconsin-Madison, USA)
\item Jack Dongarra (University of Tennessee, USA)
\item Iain Duff (Rutherford Appleton Laboratory, United Kingdom)
\item Thomas Schulthess (CSCS Swiss National Supercomputing Centre, Switzerland)
\item Satoshi Matsuoka (Tokyo Institute of Technology, Japan)
\item David Abramson (University of Queensland, Australia)
\end{compactenum}
\bigskip

The JLESC will have a \textbf{Director} reporting to the Steering Committee every year. The Director will be appointed for a 2 years term by the Steering Committee. The JLESC Director is responsible for working with partner institutions to manage the JLESC activities on a day to day basis and manage the JLESC funds.  The JLESC Director will provide a yearly report of JLESC activities and accomplishments to the Steering and Scientific Advisory Committees. The current JLESC director is Franck Cappello from ANL (80\%) and UIUC (20\%) .
\bigskip

The JLESC \textbf{Executive Committee} will consist of one JLESC representative of each full partner institution. The members of the Executive Committee will be designated by the Steering Committee and the JLESC Director will be chosen among them. The JLESC representatives are responsible for overseeing their institutions' collaborative efforts and for assisting the Director in stimulating and organizing activities and in creating the annual report. 
The Executive Committee will meet periodically as necessary to carry out their responsibilities. The Steering, Scientific Advisory and Executive Committees will meet at least once every year during the annual evaluation days. The current members of the JLESC executive committee are:
\begin{compactenum}
\item Bill Kramer (UIUC/NCSA)
\item Yves Robert (Inria)
\item Franck Cappello (ANL, UIUC)
\item Jesus Labarta (BSC)
\item Robert Speck (JSC)
\item Mitsuhisa Sato (Riken)
\end{compactenum}

\subsection{Facilities} 
An institution involved in JLESC will provide to JLESC visitors involved in collaborations (if available): office space and internet access, administrative support, and make all reasonable efforts to provide them access to its local High Performance Computing resources during their visit.

\subsection{New Members} 
Organizations may apply for JLESC membership by contacting the Director and submitting a self-nomination letter explaining the value the submitting organization will bring to the JLESC and the organization's commitment to participate as described in this document.  New partners will be approved by unanimous agreement of the founding partners. 

\subsection{Partnership University Fund (PUF) NEXTGEN Project} 
We put here the text of the PUF proposal that was accepted in 2013.

The PUF project builds on the existing successful joint laboratory between Inria and UIUC that has produced in past four years many top-level publications, some of which resulted in student awards;  and several software packages that are making their way to production in Europe and USA. The PUF NEXTGEN extends the collaboration to Argonne National Laboratory (ANL) and CNRS researchers who will bring their unique expertise and their skills to help addressing the scalability issue of simulation platforms. 

The objective of this PUF proposal is to develop new research and form new generations of researchers focusing on upcoming challenging HPC environments: Exascale machines and HPC Clouds. We will address the main obstacles to scalable simulation platforms by 1) reducing dramatically, during the simulation, the communication cost between the computational cores, the memory, the network and I/O,  thus reducing power consumption; and 2) improving the reliability of the system to make sure that scalable simulations complete with correct results. Typically, we seek gains of one order of magnitude in these two areas as compared to the current situation.

We propose a set of activities designed to establish and develop the collaboration between researchers of ANL, Inria, UIUC and CNRS and the education of young researchers: A) an unprecedented volume of students and faculty visits in support of joint research, B) two research workshops per year (one in France and one in USA) with strong student participation and C) one summer school per year opened to the students of the collaboration. We will also initiate discussions on the establishment of dual degrees between UIUC and several Universities in France.

Our project focuses on new challenging HPC environments which are exascale systems and HPC clouds. In particular we focus on their common hardware and software characteristics: system is composed of nodes communicating via a communication network. Nodes themselves use many core technologies with shared memory. Application data are stored on Input/output devices composed of storage components connected to the communication network. The very large system is subject to frequent failures (several per day) that lead to application crashes. These upcoming HPC environments also share the same programming models for HPC applications. We will focus on the Message Programming Interface (MPI) that is extremely popular because of its rich interface and the performance levels reached by available implementations. Most of the numerical simulation codes currently running in HPC centers use MPI and MPI is considered the main programming interface for Exascale systems and HPC clouds.

%The scalability of numerical simulation platforms on these environments depends on an efficient coupling between the MPI simulation codes and the characteristics of these environments. In particular two main challenges need to be solved: 1) Communication within nodes, between nodes and to I/O devices has to be minimized in order to maximize the simulation performance and reduce power usage. 2) Fault tolerance techniques need to be improved so that they can guarantee correct completion of computations without entailing unacceptable overheads. 

%We address these two challenges with seven efforts: 1) Designing new algorithms with reduced communication; 2) Reducing the internode communication overhead for the remaining communication; 3) Reducing I/O overheads; 4) Improving fault tolerance techniques to reduce their overhead; 5) Verifying numerical simulation correctness and quantifying uncertainties; 6) Improving the performance of numerical simulations in a Cloud environment; and 7) Improving large scale performance evaluation, modeling and simulation.

The PUF funding (2014-2016) complements the JLPC funding to cover workshops, summer schools and visits between Inria, Argonne, UIUC and CNRS partners of the PUF NextGN project. The new JLESC laboratory extends the JLPC by adding BSC, JSC and Riken/AICS as a new partners. While BSC students and professors can attend the PUF NextGN summer school, their stay cannot be covered by the PUF project.

\newpage
\section{Scientific Activities}
\label{sec.science}

\subsubsection*{Activity reports}

The next pages present the scientific activities conducted between July 2014 and June 2015.
Scientific activities are presented as activity reports of on-going collaborative projects.
Each report features 8 parts: 
\begin{compactenum}
\item Participants
\item Research topic and goals
\item Results for current year
\item Visits and meetings (done and planned)
\item Impact and publications
\item Person-Month efforts
\item Future plans
\item References
\end{compactenum}

\subsubsection*{Scientific collaborations}

Scientific collaborations cover the three main topics of the joint-laboratory: parallel programming, resilience and numerical libraries. Other activities have been started on system issues and, in particular, related to file systems, archive and graph partitioning.
Scientific collaborations are categorized under five types:
\begin{itemize}
\item Ongoing collaboration (\ongoing): Collaborative research is active, visits have been made, there have been several meetings and for some activities results have been produced and impact is significant.
\item Starting collaboration (\starting): Collaborators from both sides have been identified and collaborative research has just started. Visits have been made or are planned. 
%\item Finished collaboration (\finished): The topic behind the collaboration has been explored and produced some results. Collaborators decided to continue in other directions. 
\item Blocked/stopped collaboration (\blocked): Such collaboration has not progressed over the past year some for reasons associated with the availability of software, data or people. Ultimately, a blocked collaboration is stopped if no progress can be made.
\item Collaboration under exploration (\explore): Collaboration is not yet established. However, activity has started and gave some impact. We put in this category activities that are likely to become collaborative research.
\end{itemize}

Following the recommendations of the 2014 evaluation, we have consolidated the report by removing "finished" and "blocked activities". So the current report only has "ongoing" and "starting" activities. 


\newpage
\input{apps}
\input{architectures}
\input{numerics}
\input{perf_tools}
\input{prog_lang}
\input{resilience}
\input{storage}

\newpage
\subsection{Financial report}
\subsubsection{Expense report}
%\input{expense_report}


\end{document}
